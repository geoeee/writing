\documentclass{ctexart}

\begin{document}

题 1 汇编语言有什么样的优缺点,和机器语言相比,还有和高级语言相比,有什么额特点?

汇编语言更加贴近底层硬件,所以在执行效率上有其他高级语言无法相比的时空效率优势,但是也是正因为贴近硬件,所以编写调试困难,需要程序员了解底层硬件知识,相对高级语言比较繁琐。汇编语言相对于机器语言更加贴近人类习惯,机器语言更加难读写,更加难以调试。
~\\

题 2 什么场合使用汇编语言更加合适?

对于性能要求很高,或者十分贴近底层硬件的场合,或者没有办法使用高级语言的场合。
~\\ 

题 3 计算机系统中如何表示西文字符和汉字字符?
计算机内的字符表示,一开始,西文字符使用ASCII编码,汉字字符使用GB2132标准编码,现在已经有UTF-8,UTF-16,可以同意编码。
~\\

题 4 BCD 码是什么?

BCD, Binary Code Decimal, 二进制的十进制表示,也就是8421码。
还有使用十六进制的字符表示4位二进制,进而缩短书写长度。
~\\

题 5 字节,字和双字之间的关系?

一个字节是一个Byte,也就是8个bit,可以表示8位二进制。一个字是2个字节,也就是16位。一个双字是2个字,即32bit。
~\\

题 6 Intel的80X86家族都有哪些成员,分别有什么特点?

X86家族的CPU有一个共同的特点,就是向前兼容。
从1971年开始8位的4004处理器面世,而后的8008。发展到16位的8086,8088,186,286。再到32位的80386,486,而后的奔腾系列。
386加上了保护模式。

\end{document}