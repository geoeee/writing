\documentclass[UTF8]{ctexart} 

\title{高考在楼下}
\begin{document} 

\maketitle
      \tableofcontents   

          \section{}
          \paragraph{}题目有点奇怪,但是也没有想到很好的表述,本来就不是个擅长文字的人吧。意思很简单,在很长的时间里面,对于高考和相关的很多事情的看法和想法都是不断的变化的,从一开始的局中人,到后来的局外人,好像楼下到楼上的在走,于是就起了这么个猎奇的题目。
          \paragraph{}这几天都在高考季节,连空气都会凝固,看电视里,媒体上,好像每一年的高考都变得越来越重要了,四年之前的这个时候,我貌似没有感受到如此的压力,可能也是我没有超常发挥的原因吧,呵呵一笑。如果我很念旧的话,我应该起一个高考范本式的题目《高考之我见》,可是我没有,感觉这个B格更高一些。
          \paragraph{}说说楼下的高考,我还在局内的时候,作为一个高中生,必定是很苦逼地在试卷,知识点,分数之间游走,当时看来,没有什么可以述说的地方。高考对于一个学生来说,那就是一场已经演练过无数遍的考试,很荣幸的我在自己的学校考试,而且是住校生,没有奔波烦恼,刚好在自己学习的楼里的教室,不可不谓主场优势。
          \paragraph{}高考真的只是三天两夜,百来道题?就像三国不是从三国开始一样,高考涵盖的是数年的生活和状态,当时的我是一个传统意义上的差生,天子弩钝就不说了,学习还不得要领,更可怕的是神经极其敏感,为人比较腼腆,不喜欢问问题。这就很可怕了,回头看,基本是死路一条,但是,摊上个不错的学校,加上这里是上海,勉强够数了。别人骂得够多了,教育公平就不说了。
          \paragraph{}姑且算是三年的生活为了那一次考试,花了一年半的时间来学习三年的知识,又花了一年半的时间来复习这些知识,当然不止,同学们已经在备战六级单词,大学数学,大学物理了,因为老师说,站得更高,忘得更远些,周末也是补习班成群,其实还是那几个老师,还是那些知识,多了一些考试技巧和灵活思路,欣欣然一个产业。三年枯燥的生活是什么支撑下来的,第一是父母之命,一般听得最多的就是“好好读书,其他什么都不用你操心。”,殷切的期望,折算成分数,注射到年轻的身体里,感觉满满的动力。第二就是环境使然吧,十六七八岁,那时候的班会就早早的明白,不读书没有出路,班主任放的民工照片来恐吓,考不上好大学就是这个下场,多么可怕!大家都在齐头并进的时候,虽然不知道自己想要什么,但是大家要得必定是稀缺的,去晚了,耽搁了就没有了。在不知道什么有效的情况下,一条走过的路可能是最保险的,于是就这么走着。
          \paragraph{}当我在搂下的时候,看高考,就是走上楼的楼梯,就在那里,那之后是一个解脱,那天之后,就不用再背讨厌的英语单词,不用再记麻烦的语文课文,不用再做无数张数学卷子,不用再搞怎么都会扣个一两分岁的物理大题。不用再面对自己认为没有意义的考试,做一些自己喜欢而且没有意义的事情。
          \paragraph{}四年前的这三天,我走过了,很平淡的,没有波澜,就像之前几百场模拟的考试一样,只是因为紧张一点点,字比平常更丑了一些。离开高中学校的时候,我没有丢掉任何一本书,一张卷子,几年之后,我一张不剩地卖得八块钱,离开的时候,我没有回头,当时的我把什么都带走了,但是感觉到什么好像落在了那里,究竟是什么呢?

\end{document}
